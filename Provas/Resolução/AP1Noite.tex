\chapter{Prova da Noite}
\section{letra a}
\begin{center}
    \begin{tabular}{|c|c|c|c|c|}
        \hline
        N & Ativo (Y) & NProcessos (X) & \(X \times Y\) & \(X^2\)  \\ \hline
        1 & 20 & 8 & 160 & 64 \\ \hline
        2 & 40 & 5 & 200 & 25 \\ \hline
        3 & 5 & 15 & 75 & 225 \\ \hline
        4 & 25 & 8 & 200 & 64 \\ \hline
        5 & 2 & 20 & 40 & 400 \\ \hline
        \(\sum\) & 92 & 56 & 675 & 778 \\ \hline
    \end{tabular}
\end{center}
\(\beta_{1}\) pode ser definido da seguinte forma:
\[\beta_{1} = \frac{n \times \sum (X_{i}Y_{i}) - \sum Y_{i}\sum X_{i}}{n \times \sum X^2 - (\sum X_{i})^2}\]
Agora podemos utilizar os valores da tabela criada para calcular \(\beta_{1}\).
\[ \beta_{1} = \frac{5 \times 675 - 92 \times 56}{5 \times 778 - (56)^2} = -2,35676 \]
\(\beta_{0}\) pode ser definido como \(\beta_{0} = \bar{Y} - \beta_{1} \times \bar{X}\), logo:
\[ \beta_{0} = \frac{92}{5} + 2,35676 \times \frac{56}{5} = 44,79576 \]

Agora que temos os valores de \(\beta_{0}\) e \(\beta_{1}\) podemos interpretar o seguinte: na ausência de processos (quando o número de processos \(= 0\)) o valor do ativo da empresa é, em média, de R\$ 44,79576 milhões de reais. O \(\beta_{1}\) nos informa que, quando o número de processos aumenta em uma unidade, o valor do ativo decresce, em média, em R\$ 2,35676 milhões de reais.

\section{letra b}
A variável setor é uma variável qualitativa nominal, porém pode ser expressa como uma variável binária. Para tanto podemos escrever que \textit{serviço} \(= 0\) e \textit{comércio} \(= 1\). Com isso temos a seguinte tabela:

\begin{center}
    \begin{tabular}{|c|c|c|}
        \hline
         Ativo & N Processos & Setor \\ \hline
         20 & 8 & 0 \\ \hline
         40 & 5 & 0 \\ \hline
         5 & 15 & 1 \\ \hline
         25 & 8 & 1 \\ \hline
         2 & 20 & 0 \\ \hline
    \end{tabular}
\end{center}

A partir disso podemos construir a matriz \(X\).
\[ X =
    \begin{bmatrix}
        1 & 8 & 0 \\
        1 & 5 & 0 \\
        1 & 15 & 1 \\
        1 & 8 & 1 \\
        1 & 20 & 0
    \end{bmatrix}
\]
Também podemos construir a matriz \(Y\).
\[ Y =
    \begin{bmatrix}
        20 \\
        40 \\
        5 \\
        25 \\
        2
    \end{bmatrix}
\]

Para proceder ao cálculo de \((X'X)^{-1}\) primeiro precisamos escrever \(X'\).

\[
    X' =
    \begin{bmatrix}
        1 & 1 & 1 & 1 & 1 \\
        8 & 5 & 15 & 8 & 20 \\
        0 & 0 & 1 & 1 & 0
    \end{bmatrix}
\]
Logo, multiplicando \(X'\) por \(X\), obtemos:

\[
    X'X =
    \begin{bmatrix}
        5 & 56 & 2 \\
        56 & 778 & 23 \\
        2 & 23 & 2
    \end{bmatrix}
\]

Invertendo a matriz \(X'X\) (aqui também podemos utilizar as frações obtidas) obtemos:

\[
    (X'X)^{-1} =
    \begin{bmatrix}
        1,13732 & -0,07390 & -0,29679 \\
        -0,07390 & 0,00664 & -0,00332 \\
        -0,29679 & -0,00332 & 0,83499
    \end{bmatrix}
\]

Multiplicamos agora a matriz \(X'\) pela matriz \(Y\):
\[
    X'Y =
    \begin{bmatrix}
        92 \\
        675 \\
        30
    \end{bmatrix}
\]
 Por fim obtemos o produto de \((X'X)^{-1}\) por \(X'Y\):
 \[
    (X'X)^{-1}X'Y = \beta =
    \begin{bmatrix}
        46,39424 \\
        -2,33887 \\
        -4,49723
    \end{bmatrix} =
    \begin{bmatrix}
        \beta_{0} \\
        \beta_{1} \\
        \beta_{2}
    \end{bmatrix}
 \]

Após calculados os valores de \(\beta\) podemos interpretar que: para empresas do setor de \textit{serviços} e que não possuem nenhum processo, o valor do ativo é, em média, de R\$ 46,39424 \((\beta_{0})\) milhões de reais. Quando o número de processos aumenta em 1 unidade, o valor do ativo diminui, em média, em R\$ 2,33887 \((\beta_{1})\) milhões de reais. As empresas do setor de \textit{comércio} têm, em média, um ativo R\$ 4,49723 \((\beta_{2})\) milhões de reais menor que os das empresas do setor de \textit{serviços}.

Para saber se existem evidências de que a hipótese de exogeneidade foi violada no modelo da letra a), podemos calcular o nosso intervalo da seguinte forma:
\[ [\beta^{*}_{1}-\beta^{*}_{1} \times 25\%;\beta^{*}_{1}+\beta^{*}_{1} \times 25\%]\]
\[\beta^{*}_{1} \times 25\% = -2,33887 \times \frac{25}{100} = -0,58472\]
\[[-2,33887 - 0,58472;-2,33887 + 0,58472]\]
\[[-2,92359;-1,75415]\]

O \(\beta_{1}\) da regressão simples é igual a \(-2,35676\) e está, portanto, dentro do intervalo definido. Logo, concluímos que \textbf{não há} evidências de que \(H_{1}\) foi violada.

\section{letra c}
Para obtermos \(R^{2}\) usaremos \(R^{2} = 1 - \frac{SQR}{SQT}\)

Sabendo que:
\[\bar{Y} = \frac{92}{5} = 18,4 \]
\[\beta_{0} = 46,39424 \]
\[\beta_{1} = -2,33887 \]
\[\beta_{2} = -4,49723 \]

Primeiro calcularemos \(SQR\)
\[SQR = \sum (Y - \widehat{Y})^{2}\]
\[ [ 20 - (\beta_{0} + \beta_{1} \times 8 + \beta_{2} \times 0)]^{2} =  59,03279 \]
\[ [ 40 - (\beta_{0} + \beta_{1} \times 5 + \beta_{2} \times 0)]^{2} =  28,09117 \]
\[ [ 5 - (\beta_{0} + \beta_{1} \times 15 + \beta_{2} \times 1)]^{2} =  3,29045 \]
\[ [ 25 - (\beta_{0} + \beta_{1} \times 8 + \beta_{2} \times 1)]^{2} =  3,29041 \]
\[ [ 2 - (\beta_{0} + \beta_{1} \times 20 + \beta_{2} \times 0)]^{2} =  5,67945 \]
\[ SQR = 99,38427\]

Agora calcularemos \(SQT\)
\[SQT = \sum (Y-\bar{Y})^{2}\]
\[ ( 20 - 18,4)^{2} = 2,56 \]
\[ ( 40 - 18,4)^{2} = 466,56 \]
\[ ( 5 - 18,4)^{2} = 179,56 \]
\[ ( 25 - 18,4)^{2} = 43,56 \]
\[ ( 2 - 18,4)^{2} = 268,96 \]
\[ SQT = 961,2\]

Logo,
\[ R^{2} = 1 - \frac{99,38427}{961,2} = 0,89660\]

Isso nos diz que 89,66\% da variabilidade de \(Y\) é explicada pelos seus regressores. Ou seja, quão boa a nossa reta de regressão se ajusta aos dados.

Além disso, não é possível que \(R^{2}\) da regressão simples seja maior que o \(R^{2}\) da regressão múltipla, pois \(R^{2}\) é uma função função que sempre cresce com a inclussão de novos regressores. Portanto, o \(R^{2}\) de uma regressão múltipla \textbf{sempre} será maior que o \(R^{2}\) de uma regressão simples. Caso queiramos comparar o poder avaliativo de dois modelos com um número distinto de regressores, precisaríamos utilizar o \(R^{2}\) ajustado.

\section{letra d}
As variáveis setor e capital aberto são variáveis qualitativas nominais, porém podem ser expressas como variáveis binárias. Para tanto podemos escrever que \textit{serviço} \(= 0\), \textit{comércio} \(= 1\), \textit{sim} \(= 0\) e \textit{não} \(= 1\). Com isso temos a seguinte tabela:
\begin{center}
    \begin{tabular}{|c|c|c|}
        \hline
         Ativo & Setor & Cap. Abe. \\ \hline
         20 & 0 & 0 \\ \hline
         40 & 0 & 1 \\ \hline
         5 & 1 & 0 \\ \hline
         25 & 1 & 1 \\ \hline
         2 & 0 & 0 \\ \hline
    \end{tabular}
\end{center}

Para que seja possível utilizar o MQO precisamos atender \(H_{1}-H_{2}\). Se considerarmos que os fatores não observáveis não estão correlacionados com as variáveis explicativas, \(H_{1}\) estará atendida. Para tanto, basta que o modelo inclua uma constante \((\beta_{0})\) e esta hipótese estará atendida.

Para verificar \(H_{2}\) precisamos mostrar que \(det(X'X) \neq 0\)

Faremos isso considerando o seguinte:
\[ X =
    \begin{bmatrix}
         1 & 0 & 0 \\
         1 & 0 & 1 \\
         1 & 1 & 0 \\
         1 & 1 & 1 \\
         1 & 0 & 0 \\
    \end{bmatrix}
\]
\[ X' =
    \begin{bmatrix}
         1 & 1 & 1 & 1 & 1 \\
         0 & 0 & 1 & 1 & 0\\
         0 & 1 & 0 & 1 & 0\\
    \end{bmatrix}
\]

Logo,
\[ X'X = 
    \begin{bmatrix}
        5 & 2 & 2 \\
        2 & 2 & 1 \\
        2 & 1 & 2
    \end{bmatrix}
\]
E seu determinante é:
\[ det(X'X) = 7\]

Portanto é possível estimar esse efeito utilizando as variáveis Setor e Capital Aberto.

\section{letra e}
Mostramos que o modelo anterior atende \(H_{1}\) e \(H_{2}\), logo, é possível construir um modelo de regressão \textbf{linear} a partir dos dados. Considerando não haver correlação entre os erros de diferentes observações e que a variância dos erros é constante, o modelo anterior atenderá ao teorema de Gauss-Markov (\(H_{1}-H_{4}\) atendidas).


