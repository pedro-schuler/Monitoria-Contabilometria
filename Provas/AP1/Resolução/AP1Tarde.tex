\chapter{Prova da Tarde}
\section*{Letra a}
\begin{center}
    \begin{tabular}{|c|c|c|c|c|}
        \hline
        N & Ativo (Y) & NProcessos (X) & \(X \times Y\) & \(X^2\)  \\ \hline
        1 & 20 & 8 & 160 & 64 \\ \hline
        2 & 40 & 5 & 200 & 25 \\ \hline
        3 & 5 & 15 & 75 & 225 \\ \hline
        4 & 25 & 8 & 200 & 64 \\ \hline
        5 & 2 & 20 & 40 & 400 \\ \hline
        \(\sum\) & 92 & 56 & 675 & 778 \\ \hline
    \end{tabular}
\end{center}
\(\beta_{1}\) pode ser definido da seguinte forma:
\[\beta_{1} = \frac{n \times \sum (X_{i}Y_{i}) - \sum Y_{i}\sum X_{i}}{n \times \sum X^2 - (\sum X_{i})^2}\]
Agora podemos utilizar os valores da tabela criada para calcular \(\beta_{1}\).
\[ \beta_{1} = \frac{5 \times 675 - 92 \times 56}{5 \times 778 - (56)^2} = -2,35676 \]
\(\beta_{0}\) pode ser definido como \(\beta_{0} = \bar{Y} - \beta_{1} \times \bar{X}\), logo:
\[ \beta_{0} = \frac{92}{5} + 2,35676 \times \frac{56}{5} = 44,79576 \]

Agora que temos os valores de \(\beta_{0}\) e \(\beta_{1}\) podemos interpretar o seguinte: na ausência de processos (quando o número de processos \(= 0\)) o valor do ativo da empresa é, em média, de R\$ 44,79576 milhões de reais. O \(\beta_{1}\) nos informa que, quando o número de processos aumenta em uma unidade, o valor do ativo decresce, em média, em R\$ 2,35676 milhões de reais.

\section*{letra b}
A variável setor é uma variável qualitativa nominal, porém pode ser expressa como uma variável binária. Para tanto podemos escrever que \textit{serviço} \(= 0\) e \textit{comércio} \(= 1\).

Com isso temos a seguinte tabela:

\begin{center}
    \begin{tabular}{|c|c|c|}
        \hline
         Ativo & N Processos & Setor \\ \hline
         20 & 8 & 0 \\ \hline
         40 & 5 & 1 \\ \hline
         5 & 15 & 0 \\ \hline
         25 & 8 & 1 \\ \hline
         2 & 20 & 0 \\ \hline
    \end{tabular}
\end{center}

A partir disso podemos construir a matriz \(X\).
\[ X =
    \begin{bmatrix}
        1 & 8 & 0 \\
        1 & 5 & 1 \\
        1 & 15 & 0 \\
        1 & 8 & 1 \\
        1 & 20 & 0
    \end{bmatrix}
\]
Também podemos construir a matriz \(Y\).
\[ Y =
    \begin{bmatrix}
        20 \\
        40 \\
        5 \\
        25 \\
        2
    \end{bmatrix}
\]

Para proceder ao cálculo de \((X'X)^{-1}\) primeiro precisamos escrever \(X'\).

\[
    X' =
    \begin{bmatrix}
        1 & 1 & 1 & 1 & 1 \\
        8 & 5 & 15 & 8 & 20 \\
        0 & 1 & 0 & 1 & 0
    \end{bmatrix}
\]
Logo, multiplicando \(X'\) por \(X\), obtemos:

\[
    X'X =
    \begin{bmatrix}
        5 & 56 & 2 \\
        56 & 778 & 13 \\
        2 & 13 & 2
    \end{bmatrix}
\]

Invertendo a matriz \(X'X\) (aqui também podemos utilizar as frações obtidas) obtemos:

\[
    (X'X)^{-1} =
    \begin{bmatrix}
        2,99568 & 0,18574 & -1,78834 \\
        0,18574 & 0,01296 & 0,10151 \\
        -1,78834 & 0,10151 & 1,62851
    \end{bmatrix}
\]

Multiplicamos agora a matriz \(X'\) pela matriz \(Y\):
\[
    X'Y =
    \begin{bmatrix}
        92 \\
        675 \\
        65
    \end{bmatrix}
\]
 Por fim obtemos o produto de \((X'X)^{-1}\) por \(X'Y\):
 \[
    (X'X)^{-1}X'Y = \beta =
    \begin{bmatrix}
        33,98272 \\
        -1,74298 \\
        9,84665
    \end{bmatrix} =
    \begin{bmatrix}
        \beta_{0} \\
        \beta_{1} \\
        \beta_{2}
    \end{bmatrix}
 \]

Após calculados os valores de \(\beta\) podemos interpretar que: para empresas do setor de \textit{serviços} e que não possuem nenhum processo, o valor do ativo é, em média, de R\$ 33,98272 \((\beta_{0})\) milhões de reais. Quando o número de processos aumenta em 1 unidade, o valor do ativo diminui, em média, em R\$ 1,74298 \((\beta_{1})\) milhões de reais. As empresas do setor de \textit{comércio} têm, em média, um ativo R\$ 9,85665 \((\beta_{2})\) milhões de reais maior que os das empresas do setor de \textit{serviços}.

Para obtermos a matriz de variância-covariância precisamos primeiro calcular \(S^{2}\). Sabemos que:
\[ \beta_{0} = 33,99272 \]
\[ \beta_{1} = -1,74298 \]
\[ \beta_{2} = 9,84665 \]

Primeiro calcularemos a soma do quadrado dos resíduos \((SQR)\).
\[ [ 20 - (\beta_{0} + \beta_{1} \times 8 + \beta_{2} \times 0)]^2 = 0,00151\]
\[ [ 40 - (\beta_{0} + \beta_{1} \times 5 + \beta_{2} \times 1)]^2 = 23,86840\]
\[ [ 5 - (\beta_{0} + \beta_{1} \times 15 + \beta_{2} \times 0)]^2 = 8,05436\]
\[ [ 25 - (\beta_{0} + \beta_{1} \times 8 + \beta_{2} \times 1)]^2 = 23,86840\]
\[ [ 2 - (\beta_{0} + \beta_{1} \times 20 + \beta_{2} \times 0)]^2 = 8,25644\]
\[ \sum u^{2} = 64,06911\]

Sabemos que:
\[ S^{2} = \frac{\sum u^{2}}{n-k}\textrm{, onde }k = 3\]
\[S^{2} = \frac{64,06911}{5 - 3} = 32,03456\]

Sabemos ainda que:
\[
    (X'X)^{-1} =
    \begin{bmatrix}
        2,99568 & 0,18574 & -1,78834 \\
        0,18574 & 0,01296 & 0,10151 \\
        -1,78834 & 0,10151 & 1,62851
    \end{bmatrix}
\]

A partir disso podemos calcular a variância de \(\beta_{1}\)
\[ Var(\beta_{1}) = S^{2} \times 0,01296 = 32,03456 \times 0,01296 = 0,4151\]

Podemos calcular o nosso intervalo a partir disso:
\[ [\beta^{*}_{1}-Var(\beta^{*}_{1});\beta^{*}_{1}+Var(\beta^{*}_{1})]\]
\[[-1,74298-0,4151;-1,74298+0,4151]\]
\[[-2,1581;-1,3278]\]

O \(\beta_{1}\) da regressão simples é igual a \(-2,35676\) e está, portanto, fora do intervalo definido. Logo, concluímos que \textbf{há} evidências de que \(H_{1}\) foi violada.

\section*{letra c}
Sabemos que \(R^{2} = 1 - \frac{SQR}{SQT}\).

E da letra b) sabemos que \(SQR = 64,06911\).

Calculemos agora \(SQT\).
\[\bar{Y} = \frac{92}{5} = 18,4\]
\[SQT = \sum (Y - \bar{Y})^{2}\]
\[(Y_{1} - \bar{Y})^2 = (20 - 18,4)^2 = 2,56\]
\[(Y_{2} - \bar{Y})^2 = (40 - 18,4)^2 = 466,56\]
\[(Y_{3} - \bar{Y})^2 = (5 - 18,4)^2 = 179,56\]
\[(Y_{4} - \bar{Y})^2 = (25 - 18,4)^2 = 43,56\]
\[(Y_{5} - \bar{Y})^2 = (2 - 18,4)^2 = 268,96\]
\[SQT = \sum (Y - \bar{Y})^{2} = 961,2\]
Portanto podemos calcular \(SQT\) da seguinte forma:
\[R^{2} = 1 - \frac{SQR}{SQT} = 1 - \frac{64,06911}{961,2} = 0,93334\]
Isso nos diz que 93,33\% da variabilidade de Y é explicada pelos seus regressores, ou seja, quão bom é o ajuste da nossa reta de regressão aos dados.

\section*{letra d}
Para atender ao Teorema de Gauss-Markov precisamos \textbf{necessariamente} atender \(H_{1}-H_{4}\). Vimos na letra b) que possivelmente  o modelo da letra a) violou \(H_{1}\). Entretanto nada sugere que o modelo da letra b) também violou \(H_{1}\). Como foi necessário assumirmos \(H_{1}-H_{2}\) para construção do modelo, e caso não haja correlação entre os erros de diferentes observações, bem como a variância dos erros deverá ser constante, podemos dizer que o modelo da letra; b) atende ao teorema de Gauss-Markov, diferentemente do modelo da letra a).

\section*{letra e}
Como o modelo possui um regressor a mais, não podemos comparar diretamente os dois modelos, pois \(R^{2}\) é uma função não decrescente em \(X\). Para tanto precisamos utilizar o \(R^{2}\) ajustado.

O \(R^{2}\) ajustado é dado por:
\[ \bar{R}^{2} = 1 - \frac{SQR}{SQT} \times \frac{n - 1}{n - k}\]
Logo,
\[\bar{R}^{2}_{multipla} = 1 - \frac{64,06911}{961,2} \times \frac{5 - 1}{5 - 3} = 0,86669\]
Como informado na questão, \(\bar{R}^{2}_{simples} = 0,8285\), portanto o modelo que melhor se ajusta aos dados é o da regressão múltipla.
