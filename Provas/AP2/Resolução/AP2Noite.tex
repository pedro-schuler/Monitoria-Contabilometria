\chapter{Questão 1}
\section*{letra a}
\begin{center}
	\begin{tabular}{|c|c|c|}
		\hline
		Y (ln R.L.) & CEO Dual & Setor \\ \hline
		9,5         & 0        & 0     \\ \hline
		9,0         & 1        & 0     \\ \hline
		10,0        & 1        & 1     \\ \hline
		9,0         & 0        & 1     \\ \hline
		10,5        & 1        & 0     \\ \hline
		11,0        & 0        & 1     \\ \hline
	\end{tabular}
\end{center}

Onde ¨não¨ é representado por 0 e ¨sim¨ por 1, e Setor é 0 para TI e 1 para Eletro.

\[(X^{\prime}X) =
	\begin{bmatrix}
		1 & 1 & 1 & 1 & 1 & 1 \\
		0 & 1 & 1 & 0 & 1 & 0 \\
		0 & 0 & 1 & 1 & 0 & 1 \\
	\end{bmatrix}
	\times
	\begin{bmatrix}
		1 & 0 & 0 \\
		1 & 1 & 0 \\
		1 & 1 & 1 \\
		1 & 0 & 1 \\
		1 & 1 & 0 \\
		1 & 0 & 1 \\
	\end{bmatrix}
	=
	\begin{bmatrix}
		6 & 3 & 3 \\
		3 & 3 & 1 \\
		3 & 1 & 3 \\
	\end{bmatrix}
\]
\[(X^{\prime}X)^{-1} = \frac{1}{det(X^{\prime}X)} \times Adj(X^{\prime}X)\]
\[det(X^{\prime}X) = 12\]
\[Adj(X^{\prime}X) =
	\begin{bmatrix}
		8  & -6 & -6 \\
		-6 & 9  & 3  \\
		-6 & 3  & 9  \\
	\end{bmatrix}
\]
\[
	(X^{\prime}Y) =
	\begin{bmatrix}
		1 & 1 & 1 & 1 & 1 & 1 \\
		0 & 1 & 1 & 0 & 1 & 0 \\
		0 & 0 & 1 & 1 & 0 & 1 \\
	\end{bmatrix}
	\times
	\begin{bmatrix}
		9,5  \\
		9,0  \\
		10,0 \\
		9,0  \\
		10,5 \\
		11,0 \\
	\end{bmatrix}
	=
	\begin{bmatrix}
		59,0 \\
		29,5 \\
		30,0 \\
	\end{bmatrix}
\]
\[
	(X^{\prime}X)^{-1}(X^{\prime}Y) =
	\frac{1}{12}
	\times
	\begin{bmatrix}
		8  & -6 & -6 \\
		-6 & 9  & 3  \\
		-6 & 3  & 9  \\
	\end{bmatrix}
	\times
	\begin{bmatrix}
		59,0 \\
		29,5 \\
		30,0 \\
	\end{bmatrix}
	=
	\frac{1}{12}
	\times
	\begin{bmatrix}
		115 \\
		1,5 \\
		4,5 \\
	\end{bmatrix}
\]
\[
	\beta \approx
	\begin{bmatrix}
		9,6 \\
		0,1 \\
		0,4 \\
	\end{bmatrix}
	=
	\begin{bmatrix}
		\beta_{0} \\
		\beta_{1} \\
		\beta_{2} \\
	\end{bmatrix}
\]

Como \(X_{1}\) é uma variável discreta temos que:
\[ [e^{0,1} - 1] \times 100 \approx 10,5\% \]
Desta forma vemos que possuir um CEO Dual provoca, em média, um aumento de 10,5\% no valor da receita líquida da empresa.

\section*{letra b}
\[T = \frac{\beta_{1} - \beta_{1}^{H_{0}}}{\sqrt{Var(\beta_{1})}}\]
\[Var(\beta_{1}) = S^{2}(X^{\prime}X)^{-1}\]
\[S^{2} = \frac{SQR}{n-k}\]

\[9,5 - (9,6 + 0,1 \times 0 + 0,4 \times 0) = -0,1\]
\[9,0 - (9,6 + 0,1 \times 1 + 0,4 \times 0) = -0,7\]
\[10,0 - (9,6 + 0,1 \times 1 + 0,4 \times 1) = -0,1\]
\[9,0 - (9,6 + 0,1 \times 0 + 0,4 \times 1) = -1,0\]
\[10,5 - (9,6 + 0,1 \times 1 + 0,4 \times 0) = 0,8\]
\[11,0 - (9,6 + 0,1 \times 0 + 0,4 \times 1) = 1,0\]

\[SQR = 0,1^{2} + 0,7^{2} + 0,1^{2} + 1,0^{2} + 0,8^{2} + 1,0^{2} = 3,15\]

\[S^{2} = \frac{3,15}{6-3} = 1,05\]

\[Var(\beta_{1}) = 1,05 \times \frac{9}{12} = 0,7875\]
\[\sqrt{Var(\beta_{1})} \approx 0,887\]
\[T = \frac{0,1 - 0}{0,887} = 0,11\]
\[|0,11| < |2,35| \]

Portanto, não é possível rejeitar a hipótese nula de que \(\beta_{1} = 0\) ao nível de confiança de 90\%. Logo, não há evidências de regressão linear entre o CEO Dual e a Receita líquida da empresa.

\section*{letra c}
\begin{center}
	\begin{tabular}{|c|c|}
		\hline
		Y (ln R.L.) & CEO Dual \\ \hline
		9,5         & 0        \\ \hline
		9,0         & 1        \\ \hline
		10,0        & 1        \\ \hline
		9,0         & 0        \\ \hline
		10,5        & 1        \\ \hline
		11,0        & 0        \\ \hline
	\end{tabular}
\end{center}
\[\beta_{0} = \frac{9,5 + 9,0 + 11,0}{3} = 9,8\bar{3}\]
\[\beta_{1} = \frac{9,0 + 10,0 + 10,5}{3} - 9,8\bar{3} = 0\]

\[\gamma_{\beta_{1}} = 0 - 0,1 = -0,1\]

Estamos interessados na variável \(\beta_{1}\), neste caso o viés da variável omitida é igual a \(0,1\) e ela está subestimada.

\section*{letra d}
\begin{center}
	\begin{tabular}{|c|c|}
		\hline
		CEO Dual & Qualidade \\ \hline
		0        & 0         \\ \hline
		1        & 1         \\ \hline
		1        & 1         \\ \hline
		0        & 0         \\ \hline
		1        & 1         \\ \hline
		0        & 0         \\ \hline
	\end{tabular}
\end{center}

Onde ¨não¨ é representado por 0 e ¨sim¨ por 1. e Qualidade é 0 para alta e 1 para baixa.

\[
	(X'X) =
	\begin{bmatrix}
		1 & 1 & 1 & 1 & 1 & 1 \\
		0 & 1 & 1 & 0 & 1 & 0 \\
		0 & 1 & 1 & 0 & 1 & 0 \\
	\end{bmatrix}
	\times
	\begin{bmatrix}
		1 & 0 & 0 \\
		1 & 1 & 1 \\
		1 & 1 & 1 \\
		1 & 0 & 0 \\
		1 & 1 & 1 \\
		1 & 0 & 0 \\
	\end{bmatrix}
	=
	\begin{bmatrix}
		6 & 3 & 3 \\
		3 & 3 & 3 \\
		3 & 3 & 3 \\
	\end{bmatrix}
\]
\[det(X'X) = 0\]
O determinante da matriz \(X^{\prime}X\) é igual a zero, logo, este é um caso de multicolinearidade perfeita.

Pode-se estender este resultado para o caso em que Qualidade é definido como 1 para alta e 0 para baixa.
\begin{center}
	\begin{tabular}{|c|c|}
		\hline
		CEO Dual & Qualidade \\ \hline
		0        & 1         \\ \hline
		1        & 0         \\ \hline
		1        & 0         \\ \hline
		0        & 1         \\ \hline
		1        & 0         \\ \hline
		0        & 1         \\ \hline
	\end{tabular}
\end{center}

Logo:

\[
	(X'X) =
	\begin{bmatrix}
		1 & 1 & 1 & 1 & 1 & 1 \\
		0 & 1 & 1 & 0 & 1 & 0 \\
		1 & 0 & 0 & 1 & 0 & 1 \\
	\end{bmatrix}
	\times
	\begin{bmatrix}
		1 & 0 & 1 \\
		1 & 1 & 0 \\
		1 & 1 & 0 \\
		1 & 0 & 1 \\
		1 & 1 & 0 \\
		1 & 0 & 1 \\
	\end{bmatrix}
	=
	\begin{bmatrix}
		6 & 3 & 3 \\
		3 & 3 & 0 \\
		3 & 0 & 3 \\
	\end{bmatrix}
\]
\[det(X'X) = 0\]



\chapter{Questão 2}
\section*{letra a}
Não sabemos se \(X_{3}\) é relevante ou não, porém sabemos que \(Cov(X_{3},X_{1}) > 0\).

Se \(X_{3}\) for relevante, então ao se omitir \(X_{3}\) teremos que: \(S^{2}\) aumentará, o viés aumentará, \(R_{j}^{2}\) diminuirá e não é possível afirmar nada sobre o comportamento de \(Var(\beta_{1})\).

Se \(X_{3}\) não for relevante, então ao se omitir \(X_{3}\) teremos que: \(S^{2}\) permanecerá inalterado, o viés permanecerá inalterado, \(R_{j}^{2}\) diminuirá e \(Var(\beta_{1})\) diminuirá.

\section*{letra b}
Sabemos que \(X_{2}\) é um regressor relevante, porém não sabemos \(Cov(X_{2},X_{1})\). Como \(X_{2}\) é relevante, sua omissão sempre implicará num aumento de \(S^{2}\).

Caso \(Cov(X_{2},X_{1}) \neq 0\), então ao se omitir \(X_{2}\) teremos que: o viés aumentará, \(R_{j}^{2}\) diminuirá e não é possível afirmar nada sobre o comportamento de \(Var(\beta_{1})\).

Caso \(Cov(X_{2},X_{1}) = 0\), então ao se omitir \(X_{2}\) teremos que: o viés permanecerá inalterado, \(R_{j}^{2}\) permanecerá inalterado e \(Var(\beta_{1})\) aumentará.
