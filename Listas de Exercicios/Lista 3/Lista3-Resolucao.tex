\chapter{Questão 1}
\section*{letra a}
\begin{center}
    \begin{tabular}{|c|c|c|c|c|c|}
        \hline
        N & Preço da Ação (Y) & Gastos dos Pares (X) & ln(Y) & \(X \times Y\) & \(X^2\)  \\ \hline
        1 & 21 & 35 & 3,045 & 106,575 & 1225\\ \hline
        2 & 23,5 & 28 & 3,157 & 88,396 & 784\\ \hline
        3 & 14,6 & 60 & 2,681 & 160,860 & 3600\\ \hline
        4 & 13,1 & 71 & 2,573 & 182,683 & 5041\\ \hline
        5 & 20 & 44 & 2,996 & 131,824 & 1936\\ \hline
        6 & 18,1 & 45 & 2,896 & 130,320 & 2025\\ \hline
        7 & 20,2 & 58 & 3,006 & 174,348 & 3364\\ \hline
        8 & 18,5 & 44 & 2,918 & 128,392 & 1936\\ \hline
        \(\sum\) & 149 & 385 & 23,272 & 1103,398 & 19911\\ \hline
    \end{tabular}
\end{center}
\(\beta_{1}\) pode ser definido da seguinte forma:
\[\beta_{1} = \frac{n \times \sum (X_{i}Y_{i}) - \sum Y_{i}\sum X_{i}}{n \times \sum X^2 - (\sum X_{i})^2}\]
Agora podemos utilizar os valores da tabela criada para calcular \(\beta_{1}\).
\[ \beta_{1} = \frac{8 \times 1103,398 - 385 \times 23,272}{8 \times 19911 - (385)^2} = -0,012 \]
\(\beta_{0}\) pode ser definido como \(\beta_{0} = \bar{Y} - \beta_{1} \times \bar{X}\), logo:
\[ \beta_{0} = \frac{23,272}{8} + 0,012 \times \frac{48,125}{8} = 3,486 \]

Agora que temos os valores de \(\beta_{0}\) e \(\beta_{1}\) podemos interpretar o seguinte: na ausência de gastos dos pares (quando o gasto dos pares \(= 0)\), o preço da ação será, em média, 3,486. O \(\beta_{1}\) nos informa que, quando o gasto dos pares aumenta em uma unidade, o preço da ação decresce, em média, \(1,2\%\).
\section*{letra b}
\section*{letra c}
\section*{letra d}
\section*{letra e}
\section*{letra f}
\section*{letra g}
