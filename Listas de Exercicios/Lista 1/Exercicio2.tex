\chapter{Primeira Lista de Exercícios}
\section*{Questão 1}
\subsection*{Letra a}
\begin{center}
	\begin{tabular}{|c|c|c|c|c|}
		\hline
		Empresa  & Retorno (Y) & Agressividade (X) & \(X \times Y\) & \(X^2\) \\ \hline
		A        & 3.0         & 25                & 75             & 625     \\ \hline
		B        & 3.5         & 28                & 98             & 784     \\ \hline
		C        & 3.6         & 35                & 126            & 1225    \\ \hline
		D        & 3.1         & 20                & 62             & 400     \\ \hline
		E        & 3.3         & 25                & 82.5           & 625     \\ \hline
		F        & 3.8         & 30                & 114            & 900     \\ \hline
		G        & 4.0         & 35                & 140            & 1225    \\ \hline
		H        & 3.1         & 25                & 77.5           & 625     \\ \hline
		I        & 3.2         & 20                & 64             & 400     \\ \hline
		J        & 3.5         & 30                & 105            & 900     \\ \hline
		\(\sum\) & 34.1        & 273               & 944            & 7709    \\ \hline
	\end{tabular}
\end{center}
\(\beta_{1}\) pode ser definido da seguinte forma:
\[\beta_{1} = \frac{n\sum (X_{i}Y_{i}) - \sum Y_{i}\sum X_{i}}{n\sum X^2 - (\sum X_{i})^2}\]
Agora podemos utilizar os valores da tabela criada para calcular \(\beta_{1}\).
\[ \beta_{1} = \frac{10 \times 944 - 273 \times 34.1}{10 \times 7709 - (273)^2} = 0.05103 \]
\(\beta_{0}\) pode ser definido como \(\beta_{0} = \bar{Y} - \beta_{1} \times \bar{X}\), logo:
\[ \beta_{0} = \frac{34.1}{10} + 0.05103 \times \frac{273}{10} = 2.01688 \]

Agora que temos os valores de \(\beta_{0}\) e \(\beta_{1}\) podemos interpretar o seguinte: quando a agressividade tributária for igual a 0, o retorno sobre o investimento das empresas será, em média, igual a 2,01688 mil reais. O \(\beta_{1}\) nos informa que, quando a agressividade tributária aumenta em mil reais, o retorno sobre o investimento aumenta, em média, em 0,05103 mil reais, ou seja, aumenta em R\$ 51,03.

\subsection*{Letra b}
A variância do erro, denotada por \(\sigma^{2}\), é um parâmetro populacional e, portanto, desconhecido. Para estimá-lo podemos utilizar os seguintes estimadores:
\[
	s^{2} = \frac{\sum \hat{u_i^2}}{n-k} \quad
	\hat{\sigma^2}= \frac{\sum \hat{u_i^2}}{n}
\]

Inicialmente precisamos calcular o somatório do quadrado dos resíduos (SQR).
\[SQR = \sum (Y - \widehat{Y})^{2}\]
\[ [ 3.0 - (\beta_{0} + \beta_{1} \times 25)]^{2} =  0.08563 \]
\[ [ 3.5 - (\beta_{0} + \beta_{1} \times 28)]^{2} =  0.00295 \]
\[ [ 3.6 - (\beta_{0} + \beta_{1} \times 35)]^{2} =  0.04118 \]
\[ [ 3.1 - (\beta_{0} + \beta_{1} \times 20)]^{2} =  0.00391 \]
\[ [ 3.3 - (\beta_{0} + \beta_{1} \times 25)]^{2} =  0.00005 \]
\[ [ 3.8 - (\beta_{0} + \beta_{1} \times 30)]^{2} =  0.06361 \]
\[ [ 4.0 - (\beta_{0} + \beta_{1} \times 35)]^{2} =  0.03884 \]
\[ [ 3.1 - (\beta_{0} + \beta_{1} \times 25)]^{2} =  0.03711 \]
\[ [ 3.2 - (\beta_{0} + \beta_{1} \times 20)]^{2} =  0.02641 \]
\[ [ 3.5 - (\beta_{0} + \beta_{1} \times 30)]^{2} =  0.00228 \]
\[ SQR = 0.30197\]

Portanto:
\[
	s^{2} = \frac{0.30197}{10-8} = 0.03775 \quad
	\hat{\sigma^2}= \frac{0.30197}{10} = 0.03020
\]

Entretanto, vamos preferir utilizar \(s^{2}\) como estimador da variância do erro, pois ele é um estimador não viesado.

\subsection*{Letra c}
Para calcularmos a variância e covariância das variáveis em questão podemos utilizar uma abordagem matricial. A matriz de variância-covariância dos estimadores é dada por:
\[VarCov = S^2 \times (X'X)^{-1}\]
Como já obtivemos \(S^2\) na questão anterior, precisamos apenas calcular \((X'X)^{-1}\).

\[(X'X) = \begin{bmatrix}
		n      & \sum X     \\
		\sum X & \sum X^{2}
	\end{bmatrix}
\]

E, portanto:
\[(X'X)^{-1} = \frac{1}{n\sum X^{2} - (\sum X)^{2}} \times \begin{bmatrix}
		\sum X^{2} & -\sum X \\
		-\sum X    & n
	\end{bmatrix}
\]

Logo, temos que:
\[
	VarCov = S^2 \times (X'X)^{-1} = 0.03775 \times \frac{1}{10 \times 7709 - (273)^{2}} \times
	\begin{bmatrix}
		7709 & -273 \\
		-273 & 10
	\end{bmatrix}
\]
\[
	VarCov =
	\begin{bmatrix}
		0.11363  & -0.00402 \\
		-0.00402 & 0.00015
	\end{bmatrix}
\]

Por fim, temos que:
\[Var(\beta_{0}) = 0,11363\]
\[Var(\beta_{1}) = 0,00015\]

\subsection*{Letra d}
A hipótese que garante o não-viés dos estimadores é a hipótese de exogeneidade (H1), que diz que \(E(u|X) = 0\).

\subsection*{Letra e}
A variância de \(Var(\beta_{0} + \beta_{1})\) é dada por:

\[Var(\beta_{0} + \beta_{1}) = Var(\beta_{0}) + Var(\beta_{1}) + 2 \times Cov(\beta_{0} \beta_{1})\]
Da letra C temos que:
\[Var(\beta_{0} + \beta_{1}) = 0,11363 + 0,00015 + 2 \times (-0,00402)\]
\[Var(\beta_{0} + \beta_{1}) = 0,10574\]

\subsection*{Letra f}
Sabemos que \(R^{2}\) é dado por:
\[ R^{2} = 1 - \frac{SQR}{SQT} \]

E da letra b temos que \(SQR = 0,30197\). Agora precisamos calcular \(SQT\).

\[SQT = \sum (Y-\bar{Y})^{2}\]
\[ (3.0 - 3.41)^{2} =  0.1681 \]
\[ (3.5 - 3.41)^{2} =  0.0081 \]
\[ (3.6 - 3.41)^{2} =  0.0361 \]
\[ (3.1 - 3.41)^{2} =  0.0961 \]
\[ (3.3 - 3.41)^{2} =  0.0121 \]
\[ (3.8 - 3.41)^{2} =  0.1521 \]
\[ (4.0 - 3.41)^{2} =  0.3481 \]
\[ (3.1 - 3.41)^{2} =  0.0961 \]
\[ (3.2 - 3.41)^{2} =  0.0441 \]
\[ (3.5 - 3.41)^{2} =  0.0081 \]
\[SQT = 0.9690\]

Portanto: \( R^{2} = 1 - \frac{0.30197}{0.9690} = 0.68837\)

\subsection*{Letra g}
O modelo calculado foi:
\[ retorno_{i} = 2.01688 + 0.05103 \times BTD_{i} + u_{i} \]

Substituindo o valor apropriado de BTD, temos:
\[ retorno_{i} = 2.01688 + 0.05103 \times 29 = 3,49675 \]

\section*{Questão 2}
\subsection*{Letra a}
\begin{center}
	\begin{tabular}{|c|c|}
		\hline
		Preço (Y) & Receita Líquida (X) \\ \hline
		10        & 25                  \\ \hline
		15        & 30                  \\ \hline
		30        & 35                  \\ \hline
		16        & 28                  \\ \hline
		20        & 25                  \\ \hline
		12        & 25                  \\ \hline
	\end{tabular}
\end{center}

Utilizaremos uma abordagem matricial para derivar os estimadores de MQO. As matrizes \(X\) e \(Y\) são dadas por:
\[ X =
	\begin{bmatrix}
		1 & 25 \\
		1 & 30 \\
		1 & 35 \\
		1 & 28 \\
		1 & 25 \\
		1 & 25
	\end{bmatrix} \quad
	Y =
	\begin{bmatrix}
		10 \\
		15 \\
		30 \\
		16 \\
		20 \\
		12
	\end{bmatrix}
\]

Também podemos escrever a matrix \(X'\) da seguinte forma:
\[
	X' =
	\begin{bmatrix}
		1  & 1  & 1  & 1  & 1  & 1  \\
		25 & 30 & 35 & 28 & 25 & 25
	\end{bmatrix}
\]

Logo, multiplicando \(X'\) por \(X\), obtemos:
\[
	X'X =
	\begin{bmatrix}
		6   & 168  \\
		168 & 4784
	\end{bmatrix}
\]

Invertendo a matriz \(X'X\) obtemos:
\[
	(X'X)^{-1} =
	\frac{1}{6 \times 4784 - 168^2} \times \begin{bmatrix}
		4784 & -168 \\
		-168 & 6
	\end{bmatrix}
\]

Podemos também obter a matriz \(X'Y\), que é dada por:
\[
	X'Y =
	\begin{bmatrix}
		103 \\
		2998
	\end{bmatrix}
\]

Por fim, obtemos o produto de \((X'X)^{-1}\) por \(X'Y\):
\[
	(X'X)^{-1}X'Y = \beta =
	\frac{1}{6 \times 4784 - 168^2} \times \begin{bmatrix}
		-10912 \\
		684
	\end{bmatrix} =
	\begin{bmatrix}
		-22,73333 \\
		1,425
	\end{bmatrix} =
	\begin{bmatrix}
		\beta_{0} \\
		\beta_{1}
	\end{bmatrix}
\]

Logo, podemos interpretar que, quando a receita líquida é igual a 0, o preço do ativo é, em média, de R\$ -22,73333. Além disso, quando a receita líquida aumenta em (1) mil reais, o preço do ativo aumenta, em média, em R\$ 1,425.

\subsection*{Letra b}
A variância dos estimadores pode ser calculada utilizando a seguinte fórmula:
\[VarCov = S^2 \times (X'X)^{-1}\]

Primeiramente precisamos calcular \(S^2\). Para isso, precisamos calcular o somatório do quadrado dos resíduos (SQR).
\[SQR = \sum (Y - \widehat{Y})^{2}\]
\[ [ 10 - (\beta_{0} + \beta_{1} \times 25)]^{2} =  8.36176 \]
\[ [ 15 - (\beta_{0} + \beta_{1} \times 30)]^{2} =  25.16698 \]
\[ [ 30 - (\beta_{0} + \beta_{1} \times 35)]^{2} =  8.17005 \]
\[ [ 16 - (\beta_{0} + \beta_{1} \times 28)]^{2} =  1.36112 \]
\[ [ 20 - (\beta_{0} + \beta_{1} \times 25)]^{2} =  50.52836 \]
\[ [ 12 - (\beta_{0} + \beta_{1} \times 25)]^{2} =  0.79508 \]
\[ SQR = 94.38335\]

Logo, sabendo que \(S^2 = \frac{SQR}{n-k}\):
\[S^{2} = \frac{94.38335}{6-2} = 23.59584\]

Agora podemos calcular a matriz \(VarCov\):
\[VarCov = 23.59584 \times (X'X)^{-1}\]
\[
	VarCov =  \frac{23.59584}{6 \times 4784 - 168^2} \times
	\begin{bmatrix}
		4784 & -168 \\
		-168 & 6
	\end{bmatrix}
\]
\[
	VarCov =
	\begin{bmatrix}
		235.17187 & -8.25854 \\
		-8.25854  & 0.29495
	\end{bmatrix}
\]

Portanto, temos:
\[Var(\beta_{0}) = 235,17187\]
\[Var(\beta_{1}) = 0,29495\]

\subsection*{Letra c}
\begin{center}
	\begin{tabular}{|c|c|c|}
		\hline
		Preço (Y) & Receita Líquida (\(X_{1}\)) & Escolaridade (\(X_{2}\)) \\ \hline
		10        & 25                          & 22                       \\ \hline
		15        & 30                          & 20                       \\ \hline
		30        & 35                          & 24                       \\ \hline
		16        & 28                          & 16                       \\ \hline
		20        & 25                          & 18                       \\ \hline
		12        & 25                          & 24                       \\ \hline
	\end{tabular}
\end{center}

Por se tratar de uma regressão múltipla, precisaremos novamente adotar uma abordagem matricial. As matrizes relevantes são:
\[
	X =
	\begin{bmatrix}
		1 & 25 & 22 \\
		1 & 30 & 20 \\
		1 & 35 & 24 \\
		1 & 28 & 16 \\
		1 & 25 & 18 \\
		1 & 25 & 24
	\end{bmatrix} \quad
	Y =
	\begin{bmatrix}
		10 \\
		15 \\
		30 \\
		16 \\
		20 \\
		12
	\end{bmatrix} \quad
	X' =
	\begin{bmatrix}
		1  & 1  & 1  & 1  & 1  & 1  \\
		25 & 30 & 35 & 28 & 25 & 25 \\
		22 & 20 & 24 & 16 & 18 & 24
	\end{bmatrix}
\]

Primeiramente obtemos \(X'X\):
\[
	X'X =
	\begin{bmatrix}
		6   & 168  & 124  \\
		168 & 4784 & 3488 \\
		124 & 3488 & 2616
	\end{bmatrix}
\]
Obtemos também o determinante da matriz \(X'X\), que é igual a:
\[det(X'X) = 24064\]

Invertendo a matriz \(X'X\) obtemos:
\[(X'X)^{-1} =
	\frac{1}{24064} \times
	\begin{bmatrix}
		348800 & -6976 & -7232 \\
		-6976  & 320   & -96   \\
		-7232  & -96   & 480
	\end{bmatrix}
\]

Obtemos agora a matriz \(X'Y\):
\[
	X'Y =
	\begin{bmatrix}
		103  \\
		2998 \\
		2144
	\end{bmatrix}
\]

Podemos então calcular os coeficientes \(\beta\):
\[(X'X)^{-1}X'Y = \beta =
	\frac{1}{24064} \times
	\begin{bmatrix}
		-493056 \\
		35008   \\
		-3584
	\end{bmatrix} =
	\begin{bmatrix}
		-20.48936 \\
		1.45479   \\
		-0.14894
	\end{bmatrix} =
	\begin{bmatrix}
		\beta_{0} \\
		\beta_{1} \\
		\beta_{2}
	\end{bmatrix}
\]

\subsection*{Letra d}
A literatura sugere uma relação positiva entre a escolaridade e o preço do ativo, ou seja, espera-se que \(\beta_{2} > 0\). Entretanto, o valor obtido para \(\beta_{2}\) foi \(-0,14894\), indicando uma relação negativa entre as variáveis. Existem evidências de que a hipótese de exogeneidade foi violada, uma vez que \(\beta^{simples}_{1} \neq \beta^{múltipla}_{1}\).

\subsection*{Letra e}
Para calcularmos a variância dos estimadores, precisamos novamente calcular \(S^2\). Para isso, precisamos calcular o somatório do quadrado dos resíduos (SQR).
\[SQR = \sum (Y - \widehat{Y})^{2}\]
\[ [ 10 - (\beta_{0} + \beta_{1} \times 25 + \beta_{2} \times 22)]^{2} =  6.77930 \]
\[ [ 15 - (\beta_{0} + \beta_{1} \times 30 + \beta_{2} \times 20)]^{2} =  26.78621 \]
\[ [ 30 - (\beta_{0} + \beta_{1} \times 35 + \beta_{2} \times 24)]^{2} =  9.89901 \]
\[ [ 16 - (\beta_{0} + \beta_{1} \times 28 + \beta_{2} \times 16)]^{2} =  3.46600 \]
\[ [ 20 - (\beta_{0} + \beta_{1} \times 25 + \beta_{2} \times 18)]^{2} =  46.24721 \]
\[ [ 12 - (\beta_{0} + \beta_{1} \times 25 + \beta_{2} \times 24)]^{2} =  0.09353 \]
\[ SQR = 93.27126\]

Logo, sabendo que \(S^2 = \frac{SQR}{n-k}\):
\[S^{2} = \frac{93.27126}{6-3} = 31.09042\]

Sabendo também que \(VarCov = S^2 \times (X'X)^{-1}\), temos que:
\[VarCov = 31.09042 \times (X'X)^{-1}\]
\[VarCov =
	\frac{31.09042}{24064} \times
	\begin{bmatrix}
		348800 & -6976 & -7232 \\
		-6976  & 320   & -96   \\
		-7232  & -96   & 480
	\end{bmatrix}
\]
\[
	VarCov =
	\begin{bmatrix}
		450.64572 & -9.01291 & -9.34366 \\
		-9.01291  & 0.41344  & -0.12403 \\
		-9.34366  & -0.12403 & 0.62015
	\end{bmatrix}
\]

Portanto, temos:
\[Var(\beta_{0}) = 450,64572\]
\[Var(\beta_{1}) = 0,41344\]
\[Var(\beta_{2}) = 0,62015\]

\section*{Questão 3}
Sabemos que \(\beta_{1} = \frac{n \sum XY - \sum X \sum Y}{n \sum X^{2} - (\sum X)^{2}}\) e que \(\beta_{0} = \bar{Y} - \beta_{1} \times \bar{X}\).

Utilizando uma abordagem matricial temos as seguintes matrizes:
\[ X =
	\begin{bmatrix}
		1   & X_{1} \\
		1   & X_{2} \\
		... & ...   \\
		1   & X_{n}
	\end{bmatrix} \quad
	Y =
	\begin{bmatrix}
		Y_{1} \\
		Y_{2} \\
		...   \\
		Y_{n}
	\end{bmatrix} \quad
	X' =
	\begin{bmatrix}
		1     & 1     & ... & 1     \\
		X_{1} & X_{2} & ... & X_{n}
	\end{bmatrix}
\]

Podemos obter, então, a matriz \(X'X\):
\[  X'X =
	\begin{bmatrix}
		n      & \sum X     \\
		\sum X & \sum X^{2}
	\end{bmatrix}
\]
Invertendo a matriz \(X'X\) obtemos:
\[(X'X)^{-1} = \frac{1}{n \sum X^{2} - (\sum X)^{2}} \times
	\begin{bmatrix}
		\sum X^{2} & -\sum X \\
		-\sum X    & n
	\end{bmatrix}
\]
Obtemos também a matriz \(X'Y\):
\[  X'Y =
	\begin{bmatrix}
		\sum Y \\
		\sum XY
	\end{bmatrix}
\]

Por fim, obtemos o produto de \((X'X)^{-1}\) por \(X'Y\):
\[(X'X)^{-1}X'Y = \beta =
	\frac{1}{n \sum X^{2} - (\sum X)^{2}} \times
	\begin{bmatrix}
		\sum X^{2}\sum Y - \sum X\sum XY \\
		n\sum XY - \sum X\sum Y
	\end{bmatrix} =
	\begin{bmatrix}
		\beta_{0} \\
		\beta_{1}
	\end{bmatrix}
\]

O coeficiente \(\beta_{1}\) calculado pela abordagem matricial concorda com o valor conhecido de \(\beta_{1}\). Entretanto precisamos mostrar que o valor obtido de \(\beta_{0}\) também é equivalente ao valor conhecido. Podemos adotar uma prova por contradição, para fazer isso escreveremos que o \(\beta_{0}\) obtido pela via matricial é igual ao valor conhecido de \(\beta_{0}\).

\[ \frac{\sum X^{2}\sum Y - \sum X\sum XY}{n\sum X^{2} - (\sum X)^{2}} = \bar{Y} - \beta_{1}\bar{X1} \]

Escrevendo \(\bar{Y}\) e \(\bar{X1}\) em termos de somatórios e substituindo \(\beta_{1}\) pela sua forma conhecida, temos:


\[ \frac{\sum X^{2}\sum Y - \sum X\sum XY}{n\sum X^{2} - (\sum X)^{2}} = \frac{\sum Y}{n} - \frac{n\sum XY - \sum X\sum Y}{n\sum X^{2} - (\sum X)^{2}} \times  \frac{\sum X}{n}\]
\[ \frac{\sum X^{2}\sum Y - \sum X\sum XY}{n\sum X^{2} - (\sum X)^{2}} + \frac{n \sum XY - \sum X\sum Y}{n\sum X^{2} - (\sum X)^{2}} \times \frac{\sum X}{n} = \frac{\sum Y}{n} \]
\[ \frac{\sum X^{2}\sum Y - \sum X\sum XY}{n\sum X^{2} - (\sum X)^{2}} + \frac{\sum X(n\sum XY - \sum X\sum Y)}{n(n \sum X^{2} - (\sum X)^{2})} = \frac{\sum Y}{n} \]
\[ \frac{n(\sum X^{2}\sum Y - \sum X\sum XY) + \sum X(n\sum XY - \sum X\sum Y)}{n(n \sum X^{2} - (\sum X)^{2})} = \frac{\sum Y}{n} \]
\[ \frac{n(\sum X^{2}\sum Y - \sum X\sum XY) + n\sum X\sum XY - \sum X\sum X\sum Y}{n(n \sum X^{2} - (\sum X)^{2})} = \frac{\sum Y}{n} \]
\[ \frac{n(\sum X^{2}\sum Y - \sum X\sum XY + \sum X\sum XY) - (\sum X)^{2}  \sum Y}{n(n \sum X^{2} - (\sum X)^{2})} = \frac{\sum Y}{n} \]
\[ \frac{n\sum X^{2}\sum Y - (\sum X)^{2}  \sum Y}{n(n\sum X^{2} - (\sum X)^{2})} = \frac{\sum Y}{n} \]
\[ \frac{n\sum X^{2}\sum Y - (\sum X)^{2}  \sum Y}{n\sum X^{2} - (\sum X)^{2}} = \sum Y \]
\[ n\sum X^{2}\sum Y - (\sum X)^{2}\sum Y = \sum Y(n \sum X^{2} - (\sum X)^{2})\]
\[ n\sum X^{2}\sum Y - (\sum X)^{2}\sum Y = n\sum Y\sum X^{2} - \sum Y(\sum X)^{2}\]
\[ n\sum X^{2}\sum Y - (\sum X)^{2}\sum Y = n\sum Y\sum X^{2} - (\sum X)^{2}\sum Y\]
\[ n\sum X^{2}\sum Y = n\sum Y\sum X^{2} \]
\[ n\sum X^{2}\sum Y = n\sum X^{2}\sum Y \]
\[ n = n\]

Portanto, o valor de \(\beta_{0}\) obtido pela via matricial é equivalente ao valor conhecido de \(\beta_{0}\).

Precisamos agora mostrar que, ao atendermos H1, os valores dos coeficiente obtidos são não viesados. Para esta condição ser satisfeita precisamos que \(\mathbb{E}(\hat{\beta}) = \beta\)

Sabemos que:
\[\hat{\beta} = (X'X)^{-1}X'Y\]

E sabemos que:
\[Y = X\beta + u\]

Substituindo \(Y\) na equação de \(\hat{\beta}\), temos:
\[\hat{\beta} = (X'X)^{-1}X'(X\beta + u)\]
\[\hat{\beta} = (X'X)^{-1}X'X\beta + (X'X)^{-1}X'u\]
\[\hat{\beta} = I\beta + (X'X)^{-1}X'u\]
\[\hat{\beta} = \beta + (X'X)^{-1}X'u\]
\[\mathbb{E}(\hat{\beta} \mid X) = \mathbb{E}(\beta + (X'X)^{-1}X'u \mid X)\]
\[\mathbb{E}(\hat{\beta} \mid X) = \mathbb{E}(\beta \mid X) + \mathbb{E}((X'X)^{-1}X' \mid X) \mathbb{E}(u \mid X)\]

Pela hipótese de exogeneidade sabemos que \(\mathbb{E}(u \mid X) = 0\), logo:
\[\mathbb{E}(\hat{\beta} \mid X) = \mathbb{E}(\beta \mid X) + \mathbb{E}((X'X)^{-1}X' \mid X) \times 0\]
\[\mathbb{E}(\hat{\beta} \mid X) = \mathbb{E}(\beta \mid X)\]
\[\mathbb{E}(\hat{\beta} \mid X) = \beta\]
